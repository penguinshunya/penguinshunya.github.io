\documentclass{ujreport}
\usepackage{ascmac}
\usepackage[top=20truemm,bottom=20truemm,left=30truemm,right=30truemm]{geometry}
\usepackage{amsmath}

\begin{document}
\noindent
このゲームはNimそのものであるため、次の問題に言い換えられます。
\begin{screen}
$ 0 \le A_i \le K(1 \le i \le N) $、$ A_1 \oplus A_2 \oplus \ldots \oplus A_N = 0 $を満たす数列$A_1, A_2, \ldots, A_N$は何通りあるでしょうか。答えは非常に大きくなる可能性があるため、$10^9+7$で割った余りを答えてください。
($1 \le N \le 2000$、$0 \le K \le 10^{18}$)
\end{screen}
ここで非負整数$x$の$i$ビット目を$f(x, i)$とすると、$ A_1 \oplus A_2 \oplus \ldots \oplus A_N = 0 $という条件は、次のように言い換えられます。
$$ すべての正整数jについて、\sum_{i=1}^{N} f(A_i, j)は偶数 $$
つまり、$A_i$の2進数表記での各桁に注目し、1が偶数個になるような数列$A$が何通りあるかを求めればよいです。そこで次のDPを考えます。
\\\\
$dp[i][j]$ := (上位から$i$ビット目までを見たときに、$K$未満と確定した要素の数が$j$個のときの$A$の通り数)
\\\\
$dp[i][j]$が既に求まっているものとして遷移を考えます。ここで、$K$の2進数表記での桁数を$D$とおくことにします。
\\\\
{\bf $(1)$ $f(K, D - i) = 0$のとき}
\\\\
$N - j$個の要素については、$K$を超えてはいけないので$0$を選ぶ必要があり、$1$通りです。残りの$j$個の要素については、$0$と$1$のどちらを選択することも可能ですが、$1$の個数は偶数でなければならないため$2^{j-1}$通りです(ただし、$j=0$のときは$1$通りです)。ということで遷移は以下です。
\begin{alignat*}{1}
\left\{
\begin{array}{ll}
dp[i + 1][j] \mathrel{+}= dp[i][j] \times 2^{j-1} &(j \ge 1)\\
dp[i + 1][j] \mathrel{+}= dp[i][j] &(j = 0)
\end{array}
\right.
\end{alignat*}
{\bf $(2)$ $f(K, D - i) = 1$のとき}
\\\\
$j$個の要素については、$N - j$が奇数の場合は奇数個の$1$を、偶数の場合は偶数個の$1$を選ぶことになり、どちらも$2^{j-1}$通りです。\\\\
$N-j$個の要素については、$0$と$1$のどちらを選択することも可能です。$0$を選択する個数を$k$としたとき、$K$未満に確定する要素が$k$個増えるため、遷移先は$dp[i + 1][j + k]$になります。$N-j$個から$k$個の$0$を選ぶ方法は$_{N-j}C_k$通りあるため、最終的には次のような遷移になります。
$$dp[i + 1][j + k] \mathrel{+}= dp[i][j] \times 2^{j-1} \times _{N-j}C_k$$
ここで一つ注意すべき点があり、$j = 0$かつ$N-k$が奇数のときは、$\sum_{l=1}^{N} f(A_l, D - i)$が必ず奇数となるため0通りです。
\begin{alignat*}{1}
\left\{
\begin{array}{ll}
  dp[i + 1][j + k] \mathrel{+}= 0 &(j = 0かつN - kが奇数)\\
  dp[i + 1][j + k] \mathrel{+}= dp[i][j] \times _{N-j}C_k &(j = 0かつN - kが偶数)\\
  dp[i + 1][j + k] \mathrel{+}= dp[i][j] \times _{N-j}C_k \times 2^{j-1} &(otherwise)
\end{array}
\right.
\end{alignat*}
$dp[0][0] = 1$を初期値としてDPを行ったあと、$\sum_{i = 0}^{N} dp[D][i]$が答えになります。事前に$0 \le n, r \le N$の範囲の$2^n$と$_nC_r$を計算しておくことで、各遷移の計算量が$O(1)$となり、全体で$O(N^2\log K)$なので間に合います。
\end{document}
